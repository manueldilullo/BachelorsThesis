\section{Struttura della tesi}  \label{struttura_tesi}
Questa tesi di laurea si concentra sull'analisi di Spark NLP come soluzione distribuita per l'elaborazione del linguaggio naturale su vasti quantitativi di dati.

Il secondo capitolo discute il concetto di Natural Language Processing, a partire dallo scopo per il quale questa branca dall'intelligenza artificiale è nata, per poi passare alla descrizione degli step coinvolti nel processo NLP e le difficoltà che vengono incontrate durante lo stesso. Parte del capitolo pone l'attenzione sulle reti neurali utilizzate per problemi linguistici, percorrendo la storia dei primi approcci al problema fino ad arrivare ai percettroni e il Deep Learning.
All'interno del medesimo capitolo viene trattato il concetto di Language Modeling, focalizzando l'attenzione in particolar modo sui Word Embedding e sui modelli utilizzati durante la fase sperimentazione.

A seguire, nel terzo capitolo, si analizzano le tecnologie coinvolte in questo progetto di tesi, iniziando con una panoramica sul calcolo distribuito per poi proseguire descrivendo nel dettaglio l'ecosistema Apache Hadoop e gli ambienti Apache Spark e Spark NLP. Al termine di ciò viene presentato il sistema sul quale è stato eseguito il codice prodotto presentandone le caratteristiche l'hardware e la configurazione dell'ambiente software.

Nel quarto capitolo si tratta la fase di sperimentazione. Sono introdotti i task di \textit{Text Classification} di testi in lingua inglese e \textit{Named Entity Recognition} per la lingua inglese e per l'italiano, descrivendone il quesito che essi pongono e le soluzioni sviluppate per entrambi i problemi. Viene inoltre illustrata la struttura dei dataset su cui sono stati eseguiti i test e i risultati ottenuti elaborando questi ultimi.

Per concludere, nel capitolo finale, vengono esposte le conclusioni tratte dagli esiti della sperimentazione, valutando le potenzialità delle tecnologie messe in campo.