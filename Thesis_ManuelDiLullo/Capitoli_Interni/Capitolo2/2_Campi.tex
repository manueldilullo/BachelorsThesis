\section{Campi d'applicazione}
Nella fase di sperimentazione riportata in questa tesi, come già citato, sono state proposte strategie risolutive ai problemi di \textit{Classificazione del testo} e \textit{Riconoscimento delle entità nominate}. L'attenzione è stata focalizzata su questi problemi poiché su di essi si fondano numerevoli sistemi software che, ormai, vengono utilizzati da quasi la totalità delle persone nella vita quotidiana e spesso sono impiegati anche a livello aziendale da diverse imprese. Alcuni esempi sono:\\\\
\textbf{Chatbots}\\
I chatbots sono una forma d'intelligenza artificiale programmata per interagire con gli esseri umani tanto da simulare loro stessi gli esseri umani. Capiscono la complessità del linguaggio naturale e trovano il significato reale della frase e imparano anche dalle loro conversazioni con gli umani migliorandosi con il tempo.\\\\
\\
\textbf{Completamento automatico nei motori di ricerca}\\
I motori di ricerca usano i loro enormi set di dati per analizzare ciò che i loro clienti stanno probabilmente digitando quando inseriscono determinate parole e suggeriscono le possibilità più comuni.\\\\
\textbf{Assistenti vocali}\\
Usano una complessa combinazione di riconoscimento vocale, comprensione del linguaggio naturale ed elaborazione del linguaggio naturale per capire ciò che gli esseri umani dicono e poi agire di conseguenza.\\\\
\textbf{Traduttore di lingua}\\
Questi strumenti di traduzione utilizzano anche la modellazione sequence to sequence che è una tecnica di elaborazione del linguaggio naturale. In precedenza, i traduttori di lingue usavano la traduzione automatica statistica (SMT) che significava analizzare milioni di documenti già tradotti da una lingua all'altra (dall'inglese all'hindi in questo caso) e poi cercare i modelli comuni e il vocabolario di base della lingua. Tuttavia, questo metodo non era così accurato rispetto alla modellazione Sequence to sequence.\\\\
\textbf{Analisi del sentimento}\\
Le aziende possono usare la sentiment analysis per capire come un particolare tipo di utente reagisce a un particolare argomento, prodotto, ecc. Possono usare l'elaborazione del linguaggio naturale, la linguistica computazionale, l'analisi del testo, ecc. per capire il sentimento generale degli utenti per i loro prodotti e servizi e scoprire se il sentimento è buono, cattivo o neutrale.\\\\
\\
\textbf{Controlli di grammatica}\\
Non solo possono correggere la grammatica e controllare l'ortografia, ma anche suggerire sinonimi migliori e migliorare la leggibilità complessiva del contenuto.\\\\
\textbf{Classificazione e filtraggio delle e-mail}\\
I servizi di posta elettronica utilizzano l'elaborazione del linguaggio naturale per identificare il contenuto di ogni e-mail con la classificazione del testo in modo che possa essere messo nella sezione corretta. In casi più avanzati, alcune aziende utilizzano anche software antivirus speciali con elaborazione del linguaggio naturale per scansionare le e-mail e vedere se ci sono modelli e frasi che possono indicare un tentativo di phishing sui dipendenti.
