\subsection{Il Linguaggio Naturale}
Un linguaggio può essere definito come un insieme di stringhe fatte da simboli appartenenti ad un dato alfabeto. I linguaggi formali (come un linguaggio di programmazione) sono definiti con precisione: tutte le parole e il loro uso sono predefiniti nel sistema.

Il linguaggio naturale, d'altra parte, non è progettato; si evolve secondo la convenienza e l'apprendimento di un individuo. Inoltre, le macchine capiscono solo il linguaggio dei numeri, pertanto per creare modelli linguistici, è necessario convertire tutte le parole in una sequenza di numeri.

Come si determina però la “comprensione” di un linguaggio? Quando si usa questo termine si intende il capire, e quindi essere poi in grado di usare, il linguaggio a varie granularità, a partire dalle parole, in relazione al loro significato e alla appropriatezza d’uso rispetto a un contesto, fino alla grammatica e alle regole di strutturazione, sia delle frasi a partire dalle parole, sia dei paragrafi e delle pagine a partire dalle frasi.
La comprensione del linguaggio naturale però non è un compito facile, principalmente per due motivi:
\begin{enumerate}
    \item Esso deve sottostare a specifiche regole sintattiche e semantiche ma viene spesso affiancato da forme idiomatiche e convenzioni che fanno si che le frasi possano assumere un significato diverso in base al contesto nelle quali vengono utilizzate;
    \item Un'altra cosa da notare riguardo al linguaggio umano è che si tratta di simboli. Secondo Chris Manning, professore di Machine Learning a Stanford, esso è un sistema di segnalazione discreto, simbolico e categorico. Questo significa che possiamo trasmettere lo stesso significato in modi diversi (cioè, discorso, gesto, segni, ecc.). La codifica da parte del cervello umano è un modello continuo di attivazione con cui i simboli sono trasmessi attraverso segnali continui di suono e visione.
\end{enumerate}