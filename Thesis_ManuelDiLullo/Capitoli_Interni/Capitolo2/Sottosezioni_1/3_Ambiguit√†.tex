\clearpage
\subsection{Ambiguità}
L’interpretazione delle parole contestualizzate in una frase è un processo molto articolato e complesso. Ancora oggi si riscontrano ambiguità nel parlato. L’aspetto della generazione del linguaggio viene risolto invece con le regole sintattiche e sintagmatiche della lingua. L’elaborazione di testi corrisponde ad esempio a comprendere molteplici aspetti relativi al suo significato.

Un accuratezza linguistica (uso del linguaggio) viene paragonato al grado di approssimazione della performance dei parlanti nativi. Tanto più questa accuratezza viene riscontrata tanto più verrà usata ad esempio da un programma di sintesi vocale.
Molto spesso però la natura stessa del linguaggio nasconde delle ambiguità. Possono esserci quattro tipi di ambiguità (Figura \ref{fig:ambiguità} a pagina \pageref{fig:ambiguità}):): 
\begin{enumerate}
    \item \textbf{Fonologica}: l’utilizzo variabile degli accenti. Parole scritte nello stesso modo hanno spesso diverse fonologie (es. déi, dèi);
    \item \textbf{Morfologica}: derivante dalla struttura grammaticale delle parole. Stabilisce la classificazione delle parole e l'appartenenza a determinate categorie come il nome, il pronome, il verbo, l'aggettivo;
    \item \textbf{Grammaticale}: l’associazione delle parole al contesto. Analisi della struttura grammaticale dell’espressione linguistica;
    \item \textbf{Semantica}: il significato intrinseco stesso della parola. Tolte le ambiguità strutturali, ricerca l’insieme dei significati dell’espressione derivata dagli step precedenti
\end{enumerate}

\begin{figure}[hbt!]
    \centering
    \includegraphics[width=1\textwidth]{img/ambiguità.png}
    \caption{Livelli di ambiguità}
    \label{fig:ambiguità}
\end{figure}