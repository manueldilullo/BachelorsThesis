\section{Spark NLP}
La popolarità del Natural Language Processing ha fatto si che, negli ultimi anni, siano state sviluppate decine di librerie che risolvono numerosi problemi legati all'argomento (NLTK, SpaCy, Stanford Core NLP e molte altre). In particolare, negli ultimi anni ha attirato particolarmente l'attenzione Spark NLP\footnote{Spark NLP: \href{https://nlp.johnsnowlabs.com/}{https://nlp.johnsnowlabs.com/}}, libreria costruita su Apache Spark e Spark ML. Spark NLP, nata nell'ottobre del 2017 dal \textit{John Snow Labs}, è al momento la soluzione adottata maggiormente dalle aziende, grazie soprattutto alle caratteristiche ereditate da Spark che fanno si che essa ottenga i risultati migliori a livello di accuratezza e velocità di esecuzione su numerosissimi task NLP. Proprio per questo motivo, in questa Tesi, si sono volute testare in prima persona le potenzialità di questo framework.
\subsection{Cosa è Spark NLP}
Spark NLP è una libreria di elaborazione del linguaggio naturale open source, costruita su Apache Spark e Spark ML. Fornisce un’API semplice da integrare con ML Pipelines ed è commercialmente supportato da John Snow Labs.  Gli annotatori di Spark NLP utilizzano algoritmi basati su regole, Machine Learning e Deep Learning. La libreria Spark NLP è scritta in Scala e include API Scala, Java e Python per l’uso da Spark. Copre quasi la totalità dei task NLP comuni, tra cui tokenizzazione, lemmatizzazione, part-of-speech tagging, analisi del sentimento, controllo ortografico e altro ancora attraverso numerosi componenti che raggiungono lo stato dell'arte nella maggior parte dei compiti citati. In particolare, come già citato, in questo progetto sono stati trattati, utilizzando questa libreria, Named Entity Recognition e Text Classification. 

Spark NLP possiede oltre 70 pipelines ed oltre 90 modelli pre-addestrati\footnote{Models Hub: \href{https://nlp.johnsnowlabs.com/models}{https://nlp.johnsnowlabs.com/models}}, sebbene servano da modo per avere un’idea del funzionamento della libreria e non per l’uso in produzione. Essendo un'estensione nativa di Spark ML API, la libreria offre la possibilità di allenare, personalizzare e salvare i modelli in modo che possano essere eseguiti su un cluster, su altre macchine o salvati per un secondo momento (transfer learning).
\subsection{Come funziona Spark NLP}
Spark ML fornisce un insieme di applicazioni di Machine Learning che possono essere costruite
utilizzando due componenti principali: \textbf{Estimators} e \textbf{Trasformers}. Gli Estimators hanno un metodo chiamato \verb|fit(data)| che addestra un pezzo di dati a tale applicazione. Il Transformer\footnote{Elenco Transformers: \href{https://nlp.johnsnowlabs.com/docs/en/transformers}{https://nlp.johnsnowlabs.com/docs/en/transformers}} è generalmente il risultato di un processo di addestramento e applica le modifiche al set di dati di
destinazione. Questi componenti sono stati incorporati per essere applicabili a Spark NLP. Per combinare più estimators e trasformers in un unico flusso di lavoro viene utilizzato il meccanismo delle \textit{Pipelines}. Esse permettono più trasformazioni concatenate lungo un task di Machine Learning restituendo come risultato un'\textbf{annotazione}.

Gli annotatori\footnote{Elenco annotatori: \href{https://nlp.johnsnowlabs.com/docs/en/annotators}{https://nlp.johnsnowlabs.com/docs/en/annotators}} sono la punta di diamante delle funzioni NLP in Spark NLP e sono disponibili in due forme:
\begin{itemize}
    \item \textbf{Annotator Approaches}: sono quelli che rappresentano uno Spark ML Estimator e richiedono una fase di allenamento. Hanno una funzione chiamata \verb|fit(data)| che allena un modello basato su alcuni dati. Producono il secondo tipo di annotatore che è un modello annotatore o trasformatore.
    \item \textbf{Annotator Models}: sono modelli spark o trasformatori, cioè hanno una funzione \textit{transform(data)}. Questa funzione prende come input un dataframe al quale aggiunge una nuova colonna contenente il risultato dell'annotazione corrente. Tutti i trasformatori sono additivi, il che significa che aggiungono ai dati correnti, senza mai sostituire o cancellare le informazioni precedenti
\end{itemize}
Entrambe le forme di annotatori possono essere incluse in una Pipeline. Tutti gli annotatori inclusi in una Pipeline saranno automaticamente eseguiti nell'ordine definito e trasformeranno i dati di conseguenza. Una Pipeline viene trasformata in un \textbf{PipelineModel} dopo la fase \verb|fit(data)|. La Pipeline può essere salvata su disco e ricaricata in qualsiasi momento.

Nel capitolo \ref{sperimentazione} verrà illustrato nel dettaglio come questo framework è stato utilizzato durante la fase di sperimentazione.
\subsection{Perchè usare Spark NLP}
Quali sono quindi i punti di forza di Spark NLP?
\begin{enumerate}
    \item \textbf{Accuratezza}: La libreria Spark NLP 2.0 ha ottenuto i migliori risultati accademici peer-reviewed.
    \item \textbf{Velocità}: Le ragioni della sua velocità sono il motore Tungsten di seconda generazione per i dati colonnari vettoriali in-memoria, nessuna copia del testo in memoria, ampia profilazione, configurazione e ottimizzazione del codice di Spark e TensorFlow, e ottimizzazione per l'addestramento e l'inferenza.
    \item \textbf{Scalabilità}: Questa libreria è capace di scalare allenamento dei modelli, inferenza e pipelines da una macchina locale ad un cluster con piccoli, se non nessun, cambiamenti di codice.
    \item \textbf{Performance}: Spark NLP include caratteristiche che forniscono API Java, Scala e Python complete, supporta la formazione su GPU, supporta reti di deep learning definite dall'utente, supporta Spark nativamente, supporta Hadoop (YARN e HDFS).
    \item \textbf{API in Python, Java e Scala}: Una libreria che supporta più lingue non solo guadagna pubblico, ma permette anche di sfruttare i modelli implementati senza dover spostare i dati avanti e indietro tra gli ambienti di runtime.
\end{enumerate}

\clearpage