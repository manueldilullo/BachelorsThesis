\subsection{Spark NLP per Text Classification}
Spark NLP dispone di un annotatore chiamato \textit{ClassifierDL}, un classificatore multi-classe che utilizza una Deep Neural Network e supporta fino a 100 classi diverse. ClassifierDL richiede dei \verb|SENTENCE_EMBEDDINGS| (ovvero embedding di intere frasi) come input e restituisce una predizione sulla categoria a cui appartiene il testo.

I dataset estratti dai file in formato \verb|CSV| (descritti nel paragrafo \ref{textc_data}) vengono importati in uno \verb|Spark Dataframe| (argomento trattato nel paragrafo \ref{rdd_df}).

Per ottenere gli embeddings a partire dal testo, Spark NLP mette a dispozione diverse alternative; i due componenti utilizzati in questo progetto sono:
\begin{itemize}
    \item \textbf{Universal Sentence Encoder}
    \item \textbf{Bert Sentence Embeddings}
\end{itemize}
Non è il nostro scopo studiare come avviene l’addestramento dei modelli utilizzati per la produzione degli embeddings, pertanto verrano utilizzati modelli pre-addestrati presenti nella sezione 
\textit{NLP Models Hub}\footnote{Models Hub: \href{https://nlp.johnsnowlabs.com/models}{https://nlp.johnsnowlabs.com/models}} 
della documentazione di Spark NLP. Per la precisione sono stati importati:
\begin{itemize}
    \item \verb|tfhub_use|\footnote{tfhub\_use: \href{https://nlp.johnsnowlabs.com/2020/04/17/tfhub\_use.html}{https://nlp.johnsnowlabs.com/2020/04/17/tfhub\_use.html}} per \verb|UniversalSentenceEncoder|, addestrato con una Deep Averaging Network (DAN)
    \item \verb|sent_bert_base_cased|\footnote{sent\_bert\_base\_cased\href{https://nlp.johnsnowlabs.com/2020/08/25/sent\_bert\_base\_cased.html}{https://nlp.johnsnowlabs.com/2020/08/25/sent\_bert\_base\_cased.html}} per \verb|BertSentenceEmbeddings| (24 Layers di dimensione 1024).
\end{itemize}
Le \textit{pipelines} sono costruite seguendo il seguente ordine:
\begin{enumerate}
    \item \verb|Document Assembler|: prepara i dati in un formato processabile da Spark NLP.
    \item \verb|Encoder| (USE o Bert Sentence Embeddings): aggiunge una colonna al Dataframe dove, per ogni riga viene aggiunto il vettore ottenuto codificando il testo.
    \item \verb|ClassifierDL|: data in input la colonna degli embeddings ottenuta dallo step precedente, si occupa della classificazione ed aggiunge un'ulteriore colonna nella quale viene inserita la predizione della classe associata al testo.
\end{enumerate}