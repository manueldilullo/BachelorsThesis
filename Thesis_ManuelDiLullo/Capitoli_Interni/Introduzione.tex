\chapter{Abstract} \label{abstract}
Il nostro mondo è fatto di parole. L’uso della lingua è uno dei tratti principali che distingue l’homo sapiens dalle altre specie. I nostri antenati hanno inventato le lingue molte migliaia di anni fa conseguentemente alla necessità per la società umana di svilupparsi. Scimpanzé, delfini e altri animali hanno mostrato vocaboli di centinaia di segni ma solo gli esseri umani possono comunicare in modo affidabile un numero illimitato di messaggi, qualitativamente diversi, su qualsiasi argomento usando segni discreti. 
Il nostro vivere quotidiano si basa principalmente sul modo di comunicare con le altre persone: possiamo farlo oralmente, scrivendo lettere, pubblicando opere ma soprattutto, in questo periodo storico, lo si fa sfruttando la rete. Quindi a cosa serve che i nostri agenti informatici siano in grado di elaborare i linguaggi naturali? Principalmente per comunicare con gli esseri umani e per acquisire informazioni dal linguaggio scritto.

Natural Language Processing (NLP) è un campo dell'intelligenza artificiale (AI) che permette ai computer di analizzare e comprendere il linguaggio umano, sia scritto che parlato. L'elaborazione del linguaggio naturale si serve di algoritmi informatici e intelligenza artificiale per permettere ai computer di riconoscere e rispondere alla comunicazione umana. Ciò significa che ora possiamo automatizzare l'analisi e trovare informazioni che non sapevamo nemmeno di cercare.

Per rimanere al passo col crescere dei dati, soprattutto negli ultimi anni, è cresciuta anche la necessità di rendere sempre più performanti i sistemi che si occupano della loro analisi. Non sempre basta un solo apparato, ma spesso si ricorre a quello che è chiamato \textit{calcolo distribuito}

Lo sviluppo di questa tesi ha come obiettivo quello di sperimentare le prestazioni fornite dal framework Spark NLP quando esso viene impiegato per eseguire dei compiti di elaborazione del linguaggio naturale come \textit{Text Classification} e \textit{Named Entity Recognition}. Spark NLP è un progetto, dedicato appunto alla NLP, che basa le sue fondamenta su Apache Spark, un motore multilingue per l'esecuzione di ingegneria dei dati, scienza dei dati e apprendimento automatico su macchine a nodo singolo o cluster. Verranno analizzate in particolar modo la sua scalabilità, misurando e confrontando tra loro le prestazioni ottenute utilizzando cluster di diverse dimensioni.

Verrà posta l'attenzione sul concetto di elaborazione del linguaggio naturale, analizzando gli step che compongono questo processo, i suoi campi d'applicazione e i modelli al momento maggiormente utilizzati per la language modeling.
Al fine di rendere chiaro il metodo con il quale è stata svolta la fase di sperimentazione, il progetto descriverà doverosamente le tecnologie che sono state utilizzate, entrando nel dettaglio del funzionamento di Apache Spark, Spark NLP e Hadoop. Infine, verrà proposta una panoramica sull'ambiente di lavoro utilizzato per poi focalizzarsi sui task coinvolti in questo progetto, esponendone lo scopo, i casi d'uso e discutendo i risultati ottenuti tramite l'esecuzione delle soluzioni software sui nostri sistemi.