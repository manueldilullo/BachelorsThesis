\abstract{
Il nostro mondo è fatto di parole. L’uso della lingua è uno dei tratti principali che distingue l’homo sapiens dalle altre specie. I nostri antenati hanno inventato il linguaggio naturale molte migliaia di anni fa per le necessità di una società umana in via di sviluppo. Scimpanzé, delfini e altri animali hanno mostrato vocaboli di centinaia di segni ma solo gli esseri umani possono comunicare in modo affidabile un numero illimitato di messaggi qualitativamente diversi su qualsiasi argomento usando segni discreti. 
Il nostro vivere quotidiano si basa principalmente sul nostro modo di comunicare con le altre persone. Possiamo farlo oralmente, scrivendo lettere, pubblicando opere ma soprattutto, in questo periodo storico, lo si fa sfruttando la rete. Soltanto nel 2021: sono state inviati circa 319.6 miliardi di email\textsuperscript{\cite{statista_email}}, inviati 100 miliardi di messaggi tramite WhatsApp\textsuperscript{\cite{statistics_whatsapp}}, 1.8 miliardi di persone utilizzano Facebook e 1.3 miliardi accedono alla sua app di messaggistica istantanea Facebook Messenger\textsuperscript{\cite{statistics_facebook}}.
Tutto ciò genera ogni secondo una quantità immensa di informazioni. 
Perché vogliamo che i nostri agenti informatici siano in grado di elaborare i linguaggi naturali? Principalmente per comunicare con gli esseri umani e per acquisire informazioni dal linguaggio scritto. La quantità di dati che stiamo raccogliendo a livello globale sta crescendo esponenzialmente. E mentre questo accade, il numero di analisti umani sta crescendo solo linearmente - in altre parole, noi umani semplicemente non possiamo tenere il passo.\\
Questo tesoro di dati non strutturati è così vasto che non sappiamo nemmeno cosa non sappiamo. NLP ci aiuta a creare struttura all'interno di grandi volumi di dati non strutturati. Ciò significa che ora possiamo automatizzare l'analisi e trovare informazioni che non sapevamo nemmeno di cercare.\\
\\
Natural Language Processing (NLP) è un campo dell'intelligenza artificiale (AI) che permette ai computer di analizzare e comprendere il linguaggio umano, sia scritto che parlato. L'elaborazione del linguaggio naturale impiega algoritmi informatici e intelligenza artificiale per permettere ai computer di riconoscere e rispondere alla comunicazione umana.\\
\\
Per rimanere al passo col crescere dei dati, soprattutto negli ultimi anni, è cresciuta la necessità di rendere sempre più performanti i sistemi che si occupano della loro analisi. Non sempre basta un solo sistema, ma spesso si ricorre a quello che viene chiamato calcolo distribuito. Lo sviluppo di questa tesi ha come obiettivo quello di studiare il comportamento a livello prestazionale di un sistema distribuito quando esso viene impiegato per eseguire dei compiti di elaborazione del linguaggio naturale.
Si tratterà in particolar modo il framework Apache Spark, un motore multilingue per l'esecuzione di ingegneria dei dati, scienza dei dati e apprendimento automatico su macchine a nodo singolo o cluster. Spark sfrutta il paradigma del Transfer Learning che intende pre-calcolare il modello migliore che riusciamo a sviluppare, per poi sfruttare questa conoscenza.\\
}
\endabstract